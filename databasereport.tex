%%%%%%%%%%%%%%%%%%%%%%%%%%%%%%%%%%%%%%%%%%%%%%%%%%%%%%%%%%%%%%%%%%%%%%%%%%%%%%
% Detta �r ett exempel p� ett latexdokument.
% 
% Alla dokument best�r av f�ljande delar:
%
%          \documentclass[optioner]{dokumentklass}
%            ...inst�llningar...
%          \begin{document}
%            ...text...
%          \end{document}
%
% Som ni kanske redan har f�rst�tt �r anv�nds procent (%) f�r
% kommentarer.
%%%%%%%%%%%%%%%%%%%%%%%%%%%%%%%%%%%%%%%%%%%%%%%%%%%%%%%%%%%%%%%%%%%%%%%%%%%%%%

\documentclass[a4paper]{article}

\usepackage[T1]{fontenc}                % F�r svenska bokst�ver
\usepackage[swedish]{babel}             % F�r svensk avstavning och svenska
                                        % rubriker (t ex "inneh�llsf�rteckning)
\title{Database Project}
\author{Erik Stenlund, , , }
\author{Erik Stenlund (zba10est), D11, zba10est@student.lu.se \\
		Fredrik Hagfj�ll \\
		Kit Gustavsson \\
		Olof Wahlgren }
\date{}           % Blir dagens datum om det utel�mnas

\begin{document}

\maketitle   
\thispagestyle{empty}                        % Skriver ut rubriken som vi
                                % deklarerade ovan med \title, \author
                                % och eventuellt \date
\newpage

\section{Introduction}          % Detta kommando g�r en rubrik

Ord avgr�nsas av ett eller flera blanktecken. Stycken avgr�nsas av en
eller flera tomkrader. Det har ingen betydelse hur m�nga blanktecken
eller tomrader man anv�nder. Den formatterade texten ser i alla fall
likadan ut.

Bindestreck finns i tre smaker: -, -- och ---. De anv�nds som i
f�ljande exempel:

E-teknolog, 6--8 skivor br�d om dagen.

--- Vackert v�der i dag, sa amiralen.

F�r att markera text skriver man \emph{s� h�r}. Detta blir
\textbf{fetstil}.

\subsection{Varning}             % Detta blir en underrubrik

F�ljande tecken kan man inte skriva direkt: \$  \&  \#  \%  \_  \{  and  \}.  
Som synes m�ste man skriva ett bakv�nt snedstreck framf�r dem.

\subsubsection{En rubrik till}  %Detta blir en underunderrubrik

\section{Requirements}

\section{Implementation}

\newpage
\appendix
\section{E/R-diagram} \label{App:er}

\newpage
\section{Relations} \label{App:rel}

\newpage
\section{SQL statements} \label{App:sql}


\end{document}                 % The input file ends with this command.
